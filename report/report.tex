\documentclass[sigconf, nonacm]{acmart}
\settopmatter{printfolios=true}
\settopmatter{printacmref=false}
%% PACKAGES
\usepackage{graphicx}
\usepackage{hyperref}
\usepackage{cleveref}
\usepackage{subcaption}
\usepackage{natbib}
\usepackage{mathtools}
\usepackage{xcolor}

%% COLORS
\definecolor{darkgreen}{rgb}{0,0.8,0}

%% TITLES
\title{Molecule-Text contrastive learning for retrieval systems of chemical compounds}

%% AUTHORS
\author{Balthazar Neveu}
\affiliation{%
  \institution{ENS Paris-Saclay}
  \city{Saclay}
  \country{France}
}
\email{balthazar.neveu@ens-paris-saclay.fr}



%% MAIN DOCUMENT
\begin{document}

  %% KEYWORDS
  \keywords{contrastive learning, graph neural networks, large language models}

  %% Teaser figure
  \begin{teaserfigure}
    \includegraphics[width=1.\textwidth]{figures/mol_text_overview.PNG}
    \centering
    \caption{On the left side, text descriptions are transformed into sequences of tokens of various lengths. Tokenized sequences are then embedded into a vector space using a language model encoder. Each description $T_{i}$ is transformed into a vector $t_{i}$ (\color{lime} text descriptions embeddings\color{black}). In the middle, molecules $M_{i}$ will be transformed into vector descriptors $m_{i}$ (\color{purple}{molecule embeddings}\color{black}). Each atom is first transformed into a vector using its neighboring atoms. The graph structure is kept as an undirected graphs to model the bounds between atoms. A graph neural network is then used to embed these graphs into a vector space. The molecule and text descriptions embeddings are then compared. At inference time, a text sentence $T$ will be embedded into a vector $t$ which is comparable with relevant matching molecules. The closest molecule embeddings can be proposed to the chemist. Contrastive learning is used to train such a model: supervision comes from the knowlege of the pairing between the molecule and its text description. This is shown on the right side where we compute the similarity between the molecule embedding $m_{i}$ and text embeddings $t_{i}$. The idea behing contrastive learning relies on maximizing similarity between molecules and text embeddings for the correct pair and minimized for the wrong pairs.
    }
    \label{fig:original_pipeline}
  \end{teaserfigure}
  %% TITLE
  \maketitle


  %% CONTENT
  %% tbf pas sûre de l'utilité de faire un abstract, serait plus joli en fin d'intro pour contextualiser 
This study aims at building meaningful vector representations of text descriptions and molecules. The challenge is to consistently embed these two modalities into the same vector space. The most direct application is a retrieval system for chemical compounds based on text descriptions. Good molecule representations are more general as they may be re-used in other downstream tasks (interesting chemical properties may emerge from the molecule embeddings we get). We used a pretrained Large Language Model (LLM) and a Graph Convolutional Network (GCN) to map the two modalities into a shared embedding space and trained our models using contrastive learning. Our best model achieves a LRAP score of \textbf{0.905} on the  ~\href{https://www.kaggle.com/competitions/altegrad-2023-data-challenge/leaderboard}{provided challenge test set}.

\section{Introduction}
\label{sec:intro}
Recent advances in contrastive learning have allowed creating powerful representations for various modalities which can be embedded in the same shared vector space. CLIP \cite{CLIP} learns image representations which match with text description. \textit{Can the same idea be extended to molecules and text?}

Text-2-Mol \cite{text2mol} introduced the groundings of the problem of pairing a text description with the corresponding molecule.
Text-2-Mol also came with a dataset of 33k molecule paired with a textual description, made publicly available (which is always appreciated among the research community).
We're using this dataset and the problem is restricted to using this data solely. Solving this problem requires having a model with a good understanding of natural language. We're leveraging the language capabilities inherent to pretrained Large Language Models (LLM) to create embeddings for text descriptions. We at least start with models that have good understanding of the English language. The difficulty stands in how we guide the language model to detect text features (chemical domain understanding) which will match with the molecule patterns. To achieve this, a GNN (graph neural network) is trained along the LLM to extract information from the molecule graph structure which shall be as close as possible to the text representation.
  \section{Context}
\label{sec:Context}


\subsection{Dataset}
\label{sec:dataset}
\begin{figure}
    \centering
    \includegraphics[width=1.\linewidth]{figures/molecule_visualizations.png}
    \caption{2 samples of molecules compounds paired with their description (matching correctly with their unique identifier on PubChem). These two examples show the difficulty of the task, both descriptions include the word Water being used in very different contexts. The Mol-2-Vec token ids (represented with colors in the graph) model more than atoms (molecule structures in their context neighborhood)}
    \label{fig:molecule_sample}
\end{figure}
The provided dataset is made of pairs of molecules and text descriptions. Text-2-Mol \cite{text2mol} came out with a dataset of 33 310 molecule paired with a textual description, made publicly available (which is always appreciated among the research community). Molecular structures (unitary structures slightly more generic than the atom level) have been isolated with their corresponding vector representations. Vector representation relies on Mol2Vec\cite{mol2vec} which is a machine learning algorithm to create vector representations of molecular substructures trained in an unsupervised manner over 19.9 millions of chemical compounds. In figure \ref{fig:original_pipeline}, this step corresponds to the graph features extractor step. Each node is paired with a vector of length 300 which naturally contains a lot of information about the nature of the molecular substructure (this feature representation is far more powerful than simply depicting a single atom for instance). In the Mol-2-Vec dictionary, there are 3137 tokens types (+1 unknown), each token id corresponds to a vector of dimension 300.
The provided dataset has been split into 80\% , 10\%, 10\% (train, validation, test - see table \ref{tab:dataset_stats}).

\begin{table}
    \centering
    \begin{tabular}{|c|ccc|}
    \hline
         Split &  Train &  Validation& Test\\ \hline  
         Split ratio &  80\% & 10 \% & 10\% \\ \hline  
         Graph/Text pairs&  26408&  3301& 3301\\ \hline  
         Groundtruth&  Yes&  Yes& Not public\\ \hline 
    \end{tabular}
    \caption{Dataset split}
    \label{tab:dataset_stats}
\end{table}

\subsection{State of the art}
\label{sec:sota}
In the following section, we'll briefly describe relevant work related to our topic (a more thorough study is available in Appendix \ref{sec:sota_long_version}).
CLIP \cite{CLIP} builds image representation embedded in the same vector space as textual representations and works with augmented pairs of image/text by batches of size 32000. 

Text-2-Mol \cite{text2mol} applies the same concept to molecules represented as graphs (rather than regular image grids) and uses graph convolutional networks (GCN \cite{kipfwellinggcn}) to extract meaningful information. Text-2-Mol uses tiny batches of size 32 compared to CLIP. Text-2-Mol fine tunes SciBERT (a transformer encoder \cite{scibert}) pretrained by masked language modeling on a corpus of 1 million of scientific articles. 

Molecule augmentations (atom masking, bounds deletion, subgraphs deletion) are proposed in Mol-CLR\cite{molCLR}, a contrastive learning approach applied to molecule graphs only (like SimCLR \cite{SIMCLR} or the recent DINO \cite{DINO} which work solely on images, but here transposed to molecules).
  \section{Framework overview}
\label{sec:framework}
This section describes the work which was done at software level which is definitely not a part that can be neglected when working in group on such a complex machine learning project (a lot of heavy computation power required, long training times for limited chances for improvements).  We describe the constraints we had to face and the solutions we found to overcome them. We trained nearly a hundred experiments which requires good organization and a way to track and monitor everything.
We came out with a convenient solution which allows respecting all the project constraints and allows to focus on the machine learning part of the project which will be presented in a latter section.


\subsection*{Constraints statement}
\label{sec:Constraints}
As this work was to  be done in group, we have to develop a framework so everyone could work and share their code independently.
\begin{itemize}
    \item Heterogeneous environments: several operating systems (linux, windows, macOS), various platforms (local laptop training, Kaggle Kernels notebook, remote machines at Ecole Polytechnique). On every platform, data has to be acessible and experiments results have to be stored in a way they can be retrieved. We used the Kaggle dataset feature to  host the raw dataset aswell as the preprocessed data.
    \item Privacy: as not sharing code was among the rules of the challenge, our source code remained private on GitHub (\textit{which made cloning operations even trickier when using Kaggle kernels}).
    \item Reproducibility: all our experiments are reproducible (source code tracking under git, local and cloud storage of experiments results using ~\href{https://wandb.ai/molecule-nlp-altegrad-23/molecule-nlp}{Weights and biases}).
    \item Cost: project total budget limit to 50euros. \textit{Google Collab would not have met this requirements considering the amount of experiments we did and the long training times}.
\end{itemize}


We wrote a framework which takes and solves all these constraints at once and allows to focus on training models.

\subsection*{Definition of an experiment}
An experiment is defined by a unique identifier and the instanciation of a model (tokenization method, architecture of the LLM and GNN), the configuration of an  optimizer and training hyper parameters. At inference time, we're using this unique identifier so we can safely instantiate a network and reload the weights (\textit{trained model are by construction compatible with inference. This avoids the risk of having a `.pth` weight file without knowning which architecture to use to reload it.}).
Launching the training of an experiment goes seamlessly with the following command line, below is the example for experiment 620:
\begin{verbatim}
    python train.py -e 620
\end{verbatim}

The same exact training can be ran directly on Kaggle using their API.
\begin{verbatim}
    python remote_training.py -e 620 -p -nb mol-nlp
\end{verbatim}
Finally, the evaluation mechanism generates the submission csv file and the right Kaggle API command line which can be used to submit to the hosted challenge (with a message which allows to know exactly which experiment was submitted in which conditions).

\begin{verbatim}
python evaluation.py -e 620
\end{verbatim}

\begin{figure}
    \centering
    \includegraphics[width=0.5\textwidth]{figures/training_framework.png}
    \caption{Remote training framework allows accessing free GPU resources on Kaggle Kernels. Local code is synchronized through our \textbf{private Github repository} (until the 4th of February) so the remote notebook can be run with the same code. The \textbf{Kaggle dataset} feature is used to store the raw dataset and the preprocessed data (for each tokenizer, we have stored the preprocessed dataset which saves nearly 10 minutes of computation time at the begining of each experiment). The \textbf{Weights and biases} platform is used to track the experiments results and monitor the training. Once finished, trained models can be retrieved by pulling the results.
    Note that the small piece of code to help access \href{https://github.com/balthazarneveu/mva\_pepites}{Kaggle notebooks freely from the terminal} was made available publicly (totally independantly from this challenge) to be later used on other projects.}
    \label{fig:framework}
\end{figure}

\subsection*{Training conditions}
\label{sec:training conditions}

\begin{table*}[ht]
    \centering
    \begin{tabular}{lcccc}
    \hline
    \textbf{Feature} & \textbf{Nvidia T500} & \textbf{Nvidia RTX 2060} & \textbf{Nvidia K100} & \textbf{Nvidia A4000} \\ \hline
    OS & Linux & Linux WSL & Linux in Docker & Linux \\  
    Location            & local laptop           & local laptop  & Kaggle  & Ecole Polytechnique  \\
    Access & direct & direct & Kaggle Kernels & SSH \\
    Dataset access & local SSD & local SSD & Kaggle dataset & remote download + SSD drives\\ 
    Memory           & 4 GB                        & 6 GB                           & 16 GB                          & 24 GB                          \\
    Student cost & - & Electricity $\approx$ +20 euros/month & Free & Free \\
    Availability & $\infty$ & $\infty$ & 30 hours/week 12hours/experiment & $\approx \infty$ weekends and night \\
    \hline
    \end{tabular}
    \caption{Comparison of the training platform and GPUs which were used during training}
    \label{table:gpu_comparison}
\end{table*}


To setup the training loop, we started on a single NVIDIA GeForce RTX T500 local GPU with 4Gb of RAM (\textit{the baseline provided by the challenge organizers did not even run because of memory limitation}). This was enough to make sure we could train, track and monitor progress of all our experiments and build a whole local training framework. We started by freezing the LLM weights to reduce the amount of memory required and get decent batch sizes even on the tiny T500 GPU. Due to low performances, the need to access bigger computation resources quickly came so we had to overcome this difficulty by making our framework agnostic to the training platform and still be able to recover our results and monitor from anywhere. The diversity of training and hardware platforms is shown in \ref*{table:gpu_comparison}. We considered using Google Colab Pro (datasets and models shared through Google drive) but with the consideration that we have long training times, the cost would have been prohibitive. We found a good compromise and did many experiments using Kaggle Kernels through the API which are free and provide GPU acess with 16Gb of RAM during a maximum time of 12 hours per session and up to 30hours per week.



\pagebreak

\section{Our work}
\subsection*{Preliminary study}
\label{sec:preliminary study}

\begin{table*}[h]
    \centering
    \begin{tabular}{|c|c|c|c|c|}
    \hline
    \textbf{Experiment ID} & \textbf{Model Size} & \textbf{LLM} & \textbf{GNN} & \textbf{LRAP} \\ \hline
    101         & 593k                & Frozen Distill-Bert           & Base 3 layer GCN       & 18.7\%      \\ \hline
    106         & 964k                & Frozen Distill-Bert + Adapter & Base 3 layer GCN       & 26.8\%      \\ \hline
    114         & 2.125M              & Frozen Distill-Bert + Adapter & Big 5 layer GCN        & 31.6\%      \\ \hline
    112         & 964k                & Frozen Sci-Bert + Adapter     & Base 3 layer GCN       & 36.7\%      \\ \hline
    113         & 2.125M              & Frozen Sci-Bert + Adapter     & Big 5 layer GCN        & 39.8\%      \\ \hline
    65          & 66.9M               & Trainable Bert                & Base 3 layer GCN       & 63.5\%      \\ \hline
    400         & 110M                & Trainable Sci-Bert            & Base 3 layer GCN       & 66\%        \\ \hline
    \end{tabular}
    \caption{Base Models Specifications and Performances}
    \label{tab:preliminary_study_metrics}
\end{table*}


We start with a few toy experiments to see which architecture factors are most promising (initial hope is that the performances will scale accordingly when we add all extra machine learning tricks).
\textbf{Frozen LLM weights}: We first started with by simple models based on the base GCN (3 graph convolution layers followed by a global pooling layer and 2 layers MLP). Instead of fine tuning all parameters of the LLM, we first started by freezing the LLM parameters. Although simple, this idea intuivitely has many advantages for traing:
\begin{itemize}
    \item We discard the huge memory cost of training a LLM (memory issues not only come from storing the weigths on the GPU but all the optimizers variables during backpropagation). The idea could have been pushed further by pre-computing the text embeddings and storing them on disk. 
    \item Intuivitely, freezing the LLM parameters should make the training more stable as the LLM embeddings acts as a kind of anchor that the GNN shall match.
\end{itemize}
Unfortunately, training achieves low accuracy although the number of parameters to train is lightweight. We added an "adapter" module which is simply a MLP which will adapt by projecting the text representations into a more adapted space which can match with the graph.
\textbf{Influence of the graph neural network size}: We pursued our explorations to see the impact of the GNN size. The \textbf{big GCN} (5 graph convolution layers with 2 residual connection) is more complex and has more parameters to train. We can see that the accuracy is improved by increasing the GCN size. 
\begin{itemize}
    \item Using frozen Distil-BERT: from experiment 106 (base GCN $\text{LRAP}=26.8\%$) to 114 - (big GCN $\text{LRAP}=31.6\%$)
    \item Using frozen SciBERT: from experiment 102 (base GCN $\text{LRAP}=36.7\%$) to 113 - (big GCN $\text{LRAP}=39.8\%$)
\end{itemize}


\textbf{Influence of the pretrained language model}: We also browsed Hugging Face to find models that could be dedicated to scientific-specific language processing. We foud the Sci-Bert\cite{scibert} model and could use it as a drop-in replacement for the Distil-Bert baseline model. Improvements were two-fold when changing the pretrained LLM during this preliminary study: a tokenizer dedicated to a scientific corpus seems by nature a natural choice for scientific words...here atoms and molecule names not being too frequent in common language. The Sci-Bert model may also have reasoning capabilities closer to science and chemistry reactions. This was translated by a improvement in performances. Increasing the GNN size improves accuracy.
\begin{itemize}
    \item Using the Base GCN : from experiment 106 Distil-BERT ($\text{LRAP}=26.8\%$) to experiment 112 - SciBert ($\text{LRAP}=36.7\%$).
    \item Using the Big-GCN: from experiment 114 Distil-BERT ($\text{LRAP}=31.6\%$) to experiment 113 - SciBert ($\text{LRAP}=39.8\%$).
\end{itemize}
The capacity of the network (same number of parameters) being fixed between these two experiments, it proves that the SciBERT tokenization and pretraining is definitely more suited for our task. We'd hope that this $seq 8\%$ LRAP improvement would be translated when training a fully trainable LLM.

Unfortunately we later conducted larger experiments with fully trainable LLM (starting from the pretrained weights) and the performances differences were not as good as expected. Using the Base GCN : from experiment 65 Distil-BERT ($\text{LRAP}=63.5\%$) to experiment 400 - SciBert ($\text{LRAP}=66\%$), there's not that $seq 8\%$ LRAP improvement that we had seen earler. Furthermore, it's hard to tell whether the 2.5\% improvement is due to the SciBert model pretraining being more suited for our task or the fact that the model has nearly twice as many trainable parameters than the Distil-BERT. 

This preliminary study gave us guidance that using a larger GCN and using SciBERT pretrained weights and tokenizer were good trends to follow to try improving our results. The pitfall is that fine tuning SciBERT comes with a bigger memory footprint than Distil-BERT which requires diminishing batch sizes for a fixed GPU (and as stated in the CLIP paper, large batch size seems to be one key factor of the success of contrastive learning). Experiments 65 and 400 have been cautiousy trained with batches of size 32 and same hyperparameters to get comparable results: 
\begin{itemize}
    \item the SciBERT experiment 400 was only possible using a NVIDIA RTX A4000 with 24Gb of RAM \item the Distil-BERT experiment 65 was possible on a NVIDIA Tesla T100 with 16Gb of RAM (Kaggle Kernels notebook).
\end{itemize}



\begin{figure*}[ht]
    \centering
    \includegraphics[width=1.\textwidth]{figures/preliminary_study.png}
    \caption{Training curves for the preliminary study.}
    \label{fig:preliminary_study_curves}
\end{figure*}


  \section{Discussion on the Paper}
\label{sec:discussion}
Discussion
  \section{Conclusion}
\label{sec:conclusion}

In conclusion...


  \newpage

  %% APPENDIX
  \appendix
  \section{Appendix: Experiments}
\label{sec:experiments}
\subsection{Results with LoRa}
\label{sec:lora}
\begin{figure}
\includegraphics[width=0.3\textwidth]{figures/502_514.png}
\caption{LRAP scores for experiments 502 and 514}
\label{502_fig}
\end{figure}
\begin{table*}[!]
    \centering
    \begin{tabular}{|c|c|c|c|c|c|}
    \hline
    \textbf{Experiment ID} &\textbf{Batch size}& \textbf{Model size} & \textbf{LLM} & \textbf{GNN} & \textbf{LRAP} \\ \hline
    502         &  64    &1.9M & LoRA SciBERT  & Base GCN  & 61.6\%      \\ \hline
     514         &  128    &888k & LoRA BERT  & Base GCN  & 71.8\%      \\ \hline
    \end{tabular}
    \caption{Baseline experiments with LoRA BERT and SciBERT}
    \label{tab:lora}
\end{table*}
We here compare the results of experiments 502 and 514. Their validation scores are displayed on Figure \ref{502_fig}. They were conducted with the same learning rate (3e-4) without a learning rate scheduler. Their specific parameters are detailed in Table \ref{tab:lora}. Memory restrictions keep us from increasing the batch size to 128 in experiment 502.

\subsection{Results for the experiments with different GCN architectures }
\label{sec:gcn}
We compared the influence of the depth of the GCN architecture on several very close models: 502 (base GCN), 571 (big GCN),573 (fat GCN). All elements except for the GCN and the learning scheduler are identical between these experiments. 502 has no learning scheduler which explains the oscillations. We have reason to believe that the Base GCN would have still underperformed compared to 571 and 573 even with the same scheduler. The architecture and parameters of the models are detailed in Table \ref{tab:GCN} and their LRAP scores are displayed on Figure \ref{gcn_fig}
\begin{table*}[!]
    \centering
    \begin{tabular}{|c|c|c|c|c|c|}
    \hline
    \textbf{Experiment ID} &\textbf{Batch size}& \textbf{Model size} & \textbf{LLM} & \textbf{GNN} & \textbf{LRAP} \\ \hline
    502         &  64    &1.9M & LoRA SciBERT  & Base GCN 3 layers & 61.6\%      \\ \hline
     571         &  64    &2.9M & LoRA SciBERT  & Big GCN 5 layers  & 71.6\%      \\ \hline
     573        &  64    &3.4M & LoRA SciBERT  & Fat GCN 7 layers  & 79.6\%      \\ \hline
    \end{tabular}
    \caption{Experiments with LoRa SciBERT and various GCN models}
    \label{tab:GCN}
\end{table*}
\begin{figure}
\includegraphics[width=0.3\textwidth]{figures/gcn.png}
\caption{LRAP scores for experiments 502, 571 and 573}
\label{gcn_fig}
\end{figure}

\subsection{Results for the best models trained with the contrastive loss}
\label{sec:best_old}

Out of all the experiments we led in the first part of our study with the initial contrastive loss provided with the baseline for this project, these 4 experiments are the ones that yielded the best LRAP scores. They were trained only using DistilBert in order to increase the batch size as much as possible and without LoRa which we did not feel helped us increase the batch size enough (complete architecture and parameters in Table \ref{tab:best_exp_old_loss}). All these experiments were performed using the same learning rate scheduler in "plateau" mode: initial learning rate of $3e-5$, "patience" of 8 epochs and decreasing factor of 0.8. We can see here that increasing the batch size can make up for a smaller GCN architecture and vice-versa and we obtained very similar results with all these experiments. Their scores are visible in Figure \ref{fig:old_best}.

\begin{table*}[!]
    \centering
    \begin{tabular}{|c|c|c|c|c|}
    \hline
    \textbf{Experiment ID} & \textbf{Model size} & \textbf{Batch size} & \textbf{GNN} & \textbf{LRAP} \\ \hline
    9008         & 67.9M & 164      & Bigger GCN 5 layers & 85.6\%      \\ \hline
     9009         &68.4M &  128      & Fat GCN 7 layers   & 85.9\%  \\ \hline
     9010         & 68.4M & 144      & Fat GCN 7 layers   & 86.1\%     \\ \hline
     9011        & 67.9M & 180      & Big GCN 5 layers  & 86.1\% \\ \hline

    \end{tabular}
    \caption{Successful experiments trained with the contrastive loss}
    \label{tab:best_exp_old_loss}
\end{table*}

\begin{figure}
\centering
\begin{minipage}{0.4\textwidth}
\includegraphics[width=\textwidth]{figures/9010_score.png}
\end{minipage}
\hfill
\begin{minipage}{0.4\textwidth}
\includegraphics[width=\textwidth]{figures/W&B Chart 04_02_2024 10_34_42.png}   
\end{minipage}
\caption{Score for various successful experiments trained with the contrastive loss}
\label{fig:old_best}
\end{figure}

\subsection{Results for experiments trained with the binary contrastive loss}

\subsubsection{Baseline experiments}
\label{sec:baseline_new}

We trained the baseline experiments (BERT encoder and basic 3 layer GCN) with this new binary contrastive loss.Their detailed parameters can be found in Table \ref{tab:base_new_loss}. We used the learning rate 1e-4 proposed in \cite{text2mol}. We observe that the results (displayed on Figure \ref{baseline_fig}) are much higher than the previous best result on the baseline (experiment 68, LRAP score of 66\%).
\begin{table*}[!]
    \centering
    \begin{tabular}{|c|c|c|c|c|}
    \hline
    \textbf{Experiment ID} & \textbf{Model size} & \textbf{Batch size} &\textbf{Loss} & \textbf{LRAP} \\ \hline
    18         & 67.0M & 32      & Binary contrastive  & 71.5\%      \\ \hline
     19       & 67.0M  &  64      & Binary contrastive  & 79.9\%  \\ \hline
     20        & 67.0M &  128      & Binary contrastive  & 77\%     \\ \hline
     68     &67.0M   &  128     & Contrastive  & 66.7\% \\ \hline

    \end{tabular}
    \caption{Baseline experiments with different training losses}
    \label{tab:base_new_loss}
\end{table*}

\begin{figure}
\includegraphics[width=0.3\textwidth]{figures/baseline.png}
\caption{LRAP score for baseline experiments}
\label{baseline_fig}
\end{figure}

\subsubsection{Other successful experiments}
\label{sec:best_new}
In order to get other models with high scores, we reproduced models with an architecture similar to those of experiments 9008-9011 and trained them with the binary contrastive loss. These architectures are presented in further detail in Table \ref{tab:best_new_loss}.Thanks to this loss, we were able to achieve similar results with smaller batches during training. We retained the learning scheduler (initial $lr=1e-4$) and DistilBERT encoder and played with the graph and the batch sizes to obtain high performances. The scores of all these experiments are displayed on Figure \ref{fig:best_new_loss}. 

\begin{table*}[!]
    \centering
    \begin{tabular}{|c|c|c|c|c|}
    \hline
    \textbf{Experiment ID} & \textbf{Model size} & \textbf{Batch size} &\textbf{GNN} & \textbf{LRAP} \\ \hline
    9070         & 67.9M & 128      & Bigger GCN 5 layers  & 83.6\%      \\ \hline
     9072      & 67.0M  &  128      & Base GCN 3 layers  & 79.9\%  \\ \hline
     9073        & 67.9M &  64      & Bigger GCN 5 layers  & 83.0\%     \\ \hline
     9075     &68.4M   &  64     & Fat GCN 7 layers  & 86.1\% \\ \hline
      9077     & 68.4M  &  92     & Fat GCN 7 layers  & 86.3\% \\ \hline
    \end{tabular}
    \caption{Experiments with binary contrastive training loss}
    \label{tab:best_new_loss}
\end{table*}

\begin{figure}
\centering
\begin{minipage}{0.4\textwidth}
\includegraphics[width=\textwidth]{figures/9070_lr.png}
\end{minipage}
\hfill
\begin{minipage}{0.4\textwidth}
\includegraphics[width=\textwidth]{figures/9070_score.png}   
\end{minipage}
\caption{Score and learning rate for the best experiments with various GCN and batch sizes trained with the binary contrastive loss}
\label{fig:best_new_loss}
\end{figure}

\subsection*{Freezing the LLM, training a large GCN from scratch}
\label{sec:frozen}
We reused the trained model 573 which we already presented (Lora Scibert model trained with a batch size of 64, a learning rate scheduler and the 'Fat GCN' architecture) in experiment 611. We froze the trained LLM obtained via 573 and trained 611 with a bigger batch size. The details about these experiments can be found in Table \ref{tab:frozen_573} and their respective scores are displayed on Figure \ref{fig:611}. This technique enabled us to improve the LRAP score of 80\%  we had already obtained when training experiment 573. 
\begin{table*}[!]
    \centering
    \begin{tabular}{|c|c|c|c|c|}
    \hline
    \textbf{Experiment ID} & \textbf{Model size} & \textbf{Batch size} &\textbf{GNN} & \textbf{LRAP} \\ \hline
    573         & 3.4M & 64    & Fat GCN 7 layers  & 79.6\%      \\ \hline
     611 & 2.4 & 192 & Fat GCN 7 layers & 86.7%\ \\ \hline
    \end{tabular}
    \caption{Fully trained experiment and experiment with the frozen LLM and GCN trained from scratch}
    \label{tab:frozen_573}
\end{table*}

\begin{figure}
\centering
\begin{minipage}{0.4\textwidth}
\includegraphics[width=\textwidth]{figures/611_lr.png}
\end{minipage}
\hfill
\begin{minipage}{0.4\textwidth}
\includegraphics[width=\textwidth]{figures/611_score.png}   
\end{minipage}
\caption{Score and learning rate for fully trained experiment 573 and experiment 611 with frozen LLM}
\label{fig:611}
\end{figure}

% \end{table*}
  
  \newpage
  %% BIBLIOGRAPHY
  \bibliographystyle{ACM-Reference-Format}
  \bibliography{references}

\end{document}
